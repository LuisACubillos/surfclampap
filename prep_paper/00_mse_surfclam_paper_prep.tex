\PassOptionsToPackage{unicode=true}{hyperref} % options for packages loaded elsewhere
\PassOptionsToPackage{hyphens}{url}
%
\documentclass[12pt]{article}
\usepackage{lmodern}
\usepackage{amssymb,amsmath}
\usepackage{ifxetex,ifluatex}
\usepackage{fixltx2e} % provides \textsubscript
\ifnum 0\ifxetex 1\fi\ifluatex 1\fi=0 % if pdftex
  \usepackage[T1]{fontenc}
  \usepackage[utf8]{inputenc}
  \usepackage{textcomp} % provides euro and other symbols
\else % if luatex or xelatex
  \usepackage{unicode-math}
  \defaultfontfeatures{Ligatures=TeX,Scale=MatchLowercase}
\fi
% use upquote if available, for straight quotes in verbatim environments
\IfFileExists{upquote.sty}{\usepackage{upquote}}{}
% use microtype if available
\IfFileExists{microtype.sty}{%
\usepackage[]{microtype}
\UseMicrotypeSet[protrusion]{basicmath} % disable protrusion for tt fonts
}{}
\IfFileExists{parskip.sty}{%
\usepackage{parskip}
}{% else
\setlength{\parindent}{0pt}
\setlength{\parskip}{6pt plus 2pt minus 1pt}
}
\usepackage{hyperref}
\hypersetup{
            pdftitle={Assessing a management procedure for a benthic species with non-annual recruitment, the case of the surf clam (Mesodesma donacium, Lamarck 1888) in northern Patagonia, Chile.},
            pdfborder={0 0 0},
            breaklinks=true}
\urlstyle{same}  % don't use monospace font for urls
\usepackage[left=2.5cm,right=2.5cm,top=2.5cm,bottom=2.5cm,headheight=12pt,letterpaper]{geometry}
\usepackage{graphicx,grffile}
\makeatletter
\def\maxwidth{\ifdim\Gin@nat@width>\linewidth\linewidth\else\Gin@nat@width\fi}
\def\maxheight{\ifdim\Gin@nat@height>\textheight\textheight\else\Gin@nat@height\fi}
\makeatother
% Scale images if necessary, so that they will not overflow the page
% margins by default, and it is still possible to overwrite the defaults
% using explicit options in \includegraphics[width, height, ...]{}
\setkeys{Gin}{width=\maxwidth,height=\maxheight,keepaspectratio}
\setlength{\emergencystretch}{3em}  % prevent overfull lines
\providecommand{\tightlist}{%
  \setlength{\itemsep}{0pt}\setlength{\parskip}{0pt}}
\setcounter{secnumdepth}{0}
% Redefines (sub)paragraphs to behave more like sections
\ifx\paragraph\undefined\else
\let\oldparagraph\paragraph
\renewcommand{\paragraph}[1]{\oldparagraph{#1}\mbox{}}
\fi
\ifx\subparagraph\undefined\else
\let\oldsubparagraph\subparagraph
\renewcommand{\subparagraph}[1]{\oldsubparagraph{#1}\mbox{}}
\fi

% set default figure placement to htbp
\makeatletter
\def\fps@figure{htbp}
\makeatother

\usepackage{booktabs}
\usepackage{longtable}
\usepackage{array}
\usepackage{multirow}
\usepackage{wrapfig}
\usepackage{float}
\usepackage{colortbl}
\usepackage{pdflscape}
\usepackage{tabu}
\usepackage{threeparttable}
\usepackage{threeparttablex}
\usepackage[normalem]{ulem}
\usepackage{makecell}
\usepackage{lineno}
\usepackage{placeins}
\usepackage{authblk}

\linenumbers
\linespread{1}
\author[1,3,4]{Aldo Hernández}
\author[1,2,*]{Luis A. Cubillos}
\author[3]{Nicolás Muñoz}
\author[2,4]{Fabián J. Tapia}
\author[3]{Carlos Leal}
\affil[1]{Doctorado en Ciencias con mención en Manejo de Recursos Acuaticos Renovables, Universidad de Concepción, Concepción, Chile.}
\affil[2]{Centro de Investigación Oceanográfica COPAS Sur-Austral, Departamento de Oceanografía, Universidad de Concepción, Casilla 160-C,Concepción, Chile.}
\affil[3]{Centro de Investigación en Recursos Naturales, Holon SpA. Concepción, Chile.}
\affil[4]{Centro Interdisciplinario para la Investigación Acuícola (INCAR), Universidad de Concepción, Concepción, Chile.}
\affil[*]{Corresponding author. Email: lucubillos@udec.cl}
\linespread{1.5}
\usepackage{lineno} \usepackage{placeins} \linenumbers

\title{Assessing a management procedure for a benthic species with non-annual
recruitment, the case of the surf clam (Mesodesma donacium, Lamarck
1888) in northern Patagonia, Chile.}
\author{}
\date{\vspace{-2.5em}}

\begin{document}
\maketitle

\hypertarget{abstract}{%
\section{Abstract}\label{abstract}}

The exploitation of benthic species by artisanal fishers in coastal
management areas is expected to be sustainable under a management
procedure (MP), in which data from direct stock assessments are the main
input to estimate annual quotas. The adequacy of such an MP has not been
assessed for cases where recruitment does not occur annually. One such
case is the surf clam \emph{Mesodesma donacium} fishery in northern
Patagonia. We used data from direct stock assessments of \emph{M.
donacium} conducted in 2011-2017 at Cucao beach, to condition an
operating model for the population dynamics of this species. The current
MP consists of harvesting 25\% of the vulnerable stock biomass.
Simulations showed that the current exploitation rate of 25\% implies an
80\% probability of future collapse. Exploitation rate close to 15\% is
required to ensure the sustainability of this fishery. These results
highlight the need to review the current MP under the existence of
annual recruitment in benthic fisheries. In order to improve the MP
currently utilized in most artisanal fisheries along the Chilean coast,
and probably other regions, it is advisable to study alternative
harvest-control rules, and to take advantage of direct annual estimates
of biomass to develop integrated stock-assessment models.

\textbf{Key words}: data-poor fisheries, management strategy evaluation,
artisanal-fisher, management areas.

\hypertarget{introduction}{%
\section{Introduction}\label{introduction}}

Reproductive processes strongly influence the distribution and abundance
patterns of benthic species. These aspects are influenced mainly by
local coastal dynamics, which can transport or retain larvae near
spawning areas, modify the duration of larval development through
changes in water temperature (O'Connor et al., 2007), and affect the
distribution of adults (Bhaud, 1993; Giangrande et al., 1994; Grantham
et al., 2003; Ospina-Alvarez et al., 2020). Moreover, the recruitment of
benthic species depends on reproductive success, larval abundance and
dispersal, settlement success, and post-settlement survival under
environmental conditions that may be subject to anthropogenic effects
(Hunt and Scheibling, 1997; Ouréns et al., 2014). All of these factors
interact at different scales, inducing high levels of spatial and
temporal variability in recruitment (Pineda, 2000; Botsford, 2001;
Pineda et al., 2009). At a regional scale (i.e., 10-1000 km), changes in
geomorphology and coastal oceanographic regimes affect the advective
loss of larvae from settlement areas and, consequently, the recruitment
success of many species (Morgan et al., 2000; Lagos et al., 2008; Ebert,
2010). At smaller scales (0.1-10 km), local factors can strongly affect
nearshore larval distributions (Tapia and Pineda, 2007; Shanks and
Shearman, 2009), patterns of settlement (Pineda, 1994; Ladah et al.,
2005), or early mortality of benthic individuals (Hunt and Scheibling,
1997).

In Chile, one of the most important and commercially exploited benthic
species is the surf clam \emph{Mesodesma donacium}. This species
inhabits sandy beaches along the Chilean coast, from Arica to southern
Chiloé (18-43ºS). It forms dense aggregations that are associated with
morpho-dynamic beach features such as grain-size distribution,
steepness, and profile (Jaramillo et al., 1994). The landing records for
\emph{M. donacium} reveal boom and bust cycles, with significant spatial
and temporal fluctuations in landings, which have been described as
serial depletion in the populations distributed along the Chilean coast
(Thiel et al., 2007). Initially, in the 1960s and 1970s, harvesting for
this species was concentrated mainly in the northern region (Matamala et
al., 2008), particularly in the sandy banks of Coquimbo. During the late
1980s, banks in the southern zone near Mehuin were under significant
extraction pressure. The fishery practically disappeared in the
mid-1990s and started again in 1998 with the simultaneous harvesting of
10 banks in the southern Los Lagos Region (Rubilar et al., 2001; Stotz
et al., 2003). The high variability observed in the harvesting of surf
clam has been attributed to ENSO effects on their survival and
reproductive biology (Arntz et al., 1987; Riascos et al., 2009;
Carstensen et al., 2010; Ibarcena Fernández et al., 2019). Nevertheless,
recruitment variability could be caused by density-dependent effects
(Lima et al., 2000), and is probably associated with adult life span
(Ripley and Caswell, 2006).

In recent years, the fishery for the surf clam \emph{M. donacium} has
focused on three main sections of the Chilean coast: a) Coquimbo Region
(29-30ºS), with high inter-annual variability in landings, which are
concentrated at two main coves (Los Choros and Peñuelas, Fig. 1); b)
``Caleta Quidico'' in the Biobío Region (38ºS), where most of the
national landings were concentrated between 2001 and 2004, with a rapid
depletion of the bank after that; and, c) Los Lagos Region (42-43ºS),
with three main coves (Maullín, Mar Brava, and Cucao) accounting for
landings that increased substantially in 2009-2011, and then dropped to
reach a minimum in 2016 (Fig. 1).

In Chile, benthic fisheries are managed through an administrative system
known as ``Areas for the Management and Exploitation of Benthic
Resources'' (AMEBR) which is based on a Territorial User Rights for
Fisheries (TURF) system, in which a geographical coastal area is
allocated to artisanal-fisher organizations through temporary rights to
harvest benthic species. Fishers must provide baseline information and a
managing plan for target benthic species, derived from field surveys,
which are often conducted by private consultants. Based on these
surveys, the management agency (Undersecretariat of Fisheries and
Aquaculture, SUBPESCA) authorizes to harvest a given quota for the
target species, seeking to safeguard the ecosystem's natural recovery
(González et al., 2006; Gelcich et al., 2010; Marín and Gelcich, 2012;
Aburto et al., 2013).

In the AMEBR system, the management procedure (MP) consists of a) an
annual assessment of the standing stock, which provides estimates of
biomass and length-composition; b) estimation of a target fishing
mortality, usually the F0.1 (Deriso, 1987) by assuming a pseudo-cohort
and applying the yield-per-recruit model of Thompson and Bell (e.g.,
Doubleday and Esunge, 2011; Mildenberger et al., 2017); and c)
estimation of a total allowable catch (TAC), which must be authorized to
be harvested by SUBPESCA. This MP repeats annually without taking into
account past surveys and removals and would be inadequate to ensure the
sustainable exploitation of benthic species with non-annual recruitment,
such as the surf clam \emph{M. donacium}.

Unfortunately, one crucial weakness of the surf clam fishery in Chile is
the incomplete recording of landings at coves and landing ports.
Additionally, there is a lack of management plans that simultaneously
consider the impact of users and environmental variability on the target
species' availability, which increases the cost of maintaining
monitoring programs (CCT-B, 2014). Often, these fisheries are data-poor,
which makes it challenging to apply quantitative methods of population
assessment, such as integrated statistical catch-at-length (age)
analysis (Smith et al., 2009; Punt et al., 2011). Thus, the local
depletion of surf clam populations observed along the Chilean coast over
the past decades has not been adequately evaluated yet due to a lack of
data, which has hampered attempts to test whether local depletion was
due to fishing effects or larger-scale, oceanographically driven changes
in population dynamics.

At Cucao beach in northern Patagonia, data from direct stock assessments
of surf clam are available annually for the seven years 2011-2017.
During this period, different consultants conducted surveys and produced
estimates of abundance and annual quotas under the assumption that this
species recruits annually. Decisions on harvest limits for this area
were taken considering those annual quota estimates. In this study, we
use the 7-year data set from stock assessment surveys to implement an
operating model for the population dynamics of Mesodesma donacium at
Cucao beach. The implemented model allows for inter-annual variability
in recruitment to be simulated, to assess the management procedure
currently applied to AMEBRs, and to provide estimates of harvest rates
that could achieve sustainable exploitation given the high temporal
variability in recruitment detected for this fishery in the recent past.

\FloatBarrier

\hypertarget{materials-and-methods}{%
\section{Materials and methods}\label{materials-and-methods}}

\hypertarget{study-area-and-data-sources}{%
\subsection{Study area and data
sources}\label{study-area-and-data-sources}}

The study area is Cucao beach (24º36'S-74º08'W), located on the western
shore of Chiloé island, northern Chilean Patagonia (Fig. 1). Over the
past decade, Cucao has been one of the main harvesting areas for the
surf clam \emph{M. donacium}, with three organizations of artisanal
fishers having territorial use rights since 2015 (Fig. 1). The data were
obtained from six stock assessment surveys carried out between 2011 and
2017 (Table 1). Before 2015, the stock assessments of surf clam were
carried out to establish annual catch quotas. Since 2015, data from the
assessments became input information for harvesting surf clam under the
AMEBR management procedure.

\hypertarget{evaluation-of-the-management-procedure}{%
\subsection{Evaluation of the management
procedure}\label{evaluation-of-the-management-procedure}}

A simulation was implemented to evaluate the performance of the
management procedure (MP) for \emph{M. donacium} in Cucao. The
simulation involved the steps of the Management Strategy Evaluation
(MSE) framework (Starr et al., 1997; Cochrane et al., 1998; Punt et al.,
2016; Kell et al., 2017). In this framework, one of the steps is
conditioning an Operating Model (OM) based on data and knowledge for the
surf clam population dynamics. The OM allowed us to evaluate the MP
under uncertainty (Fig. 2), especially in terms of recruitment, which
during the studied period exhibited pulses of high recruitment followed
by years of low to nil recruitment. The OM allowed simulating the
perceived vulnerable biomass in the stock assessment surveys for a
window of 20 years into the future, along with the quota and the
realized total harvest under a constant harvest rate strategy.

The simulation modeling to evaluate the MP for surf clam consisted of
the steps described in the following sections: Section A describes the
current MP for surf clam in Cucao. Section B describes the OM that
specifies the true structure and processes modulating the surf clam
population dynamics, with emphasis on conditioning the OM to the
available data and knowledge (Kell et al., 2017). Section C describes
the phase of projecting the operating model 20 years into the future.
For each year, the OM provides a population that can be sampled in a way
similar to the stock assessment surveys carried out in the field. The
projected OM included the recruitment dynamics and its response to
fishing and environmental forcing. Section D describes the statistics
used to summarize the performance of the current and alternative
management procedures for surf clam in Cucao.

\hypertarget{section-a-the-management-procedure-for-surf-clam}{%
\subsection{Section A: The management procedure for surf
clam}\label{section-a-the-management-procedure-for-surf-clam}}

The management procedure corresponding to the Cucao AMEBR is shown in
Figure 3. A team of technicians and professional divers carry out a
stock assessment survey annually (see Table 1). The survey is designed
to provide estimates of total abundance and biomass in the surveyed
area. The estimate of biomass is size-structured, allowing the
estimation of vulnerable biomass, which is defined by surf clams larger
than 50 mm length (i.e., the minimum legal size). The stock assessment
team computes yield per recruit using a Thomson and Bell model, and then
\(F_{0.1}\) (Deriso, 1987) to compute the quota to be harvested.
Nevertheless, in practical terms, the harvest decision has resulted in a
constant Quota/Vulnerable biomass ratio of approximately 25\% (Table 1).
Thus, the current management procedure can be simplified by formulating
the following empirical harvest control rule:

\begin{enumerate}
\def\labelenumi{\arabic{enumi})}
\tightlist
\item
  \(Q_i=0.25 V_i\)
\end{enumerate}

where \(V_i\) is the survey estimate for vulnerable biomass in the i-th
year and \(Q_i\) is the quota of surf clam requested by the fishers
organizations to the centralized management agency, i.e., the
Undersecretariat of Fisheries and Aquaculture (SUBPESCA), which reviews
the technical reports and approves the harvest quotas. The management
procedure is essentially empirical since it uses the vulnerable biomass
estimated in the survey as an indicator of the surf clam status, and the
primary input to the harvest control rule (Table 1). Once SUBPESCA
approves the quota, fishers can harvest the surf clam from the
management area. At the time of harvest, catches are monitored and
logged by the Chilean National Fisheries Service (SERNAPESCA).

\hypertarget{section-b-the-operating-model}{%
\subsection{Section B: The operating
model}\label{section-b-the-operating-model}}

The operating model (OM) was conditioned to know life-history parameters
of surf clam and total biomass and population size-structure data
obtained from the direct stock-assessment surveys (Table 1 and Table 2).
The OM was based on an integrative size-structured stock assessment
model (Sullivan et al., 1990; Punt et al., 2013), expressed by

\begin{enumerate}
\def\labelenumi{\arabic{enumi})}
\setcounter{enumi}{1}
\tightlist
\item
  \(N_{i,l}=G_{l,l'} N_{i-1,l} e^{-Z_{i-1,l} }+ r_l R_i\)
\end{enumerate}

where \(N_{i,l}\) is the abundance of length-class \(l\) at the
beginning of year \(i\), \(Z\) is the instantaneous total mortality
rate, i.e., \(Z=F+M\), where \(F\) is the fishing mortality, and M is
the natural mortality rate (set equal to 0.3). \(R_i\) is recruitment,
\(r_l\) is the distribution of recruitment by length-classes, and
\(G_{l,l'}\) is a growth transition matrix described by

\begin{enumerate}
\def\labelenumi{\arabic{enumi})}
\setcounter{enumi}{2}
\tightlist
\item
  \(G_{l,l'}=\int_l^{l'} (l'-l)^{\alpha_j} e^{-(l'-l)/\beta_p } dl/\beta_p\)
\end{enumerate}

where \(l\) is the length class, and \(\alpha_j\) and \(\beta_p\) are
parameters describing a gamma probability function. Recruitment was
estimated according to:

\begin{enumerate}
\def\labelenumi{\arabic{enumi})}
\setcounter{enumi}{3}
\tightlist
\item
  \(R_i = \bar{R}e^{\epsilon_i}\)
\end{enumerate}

where \(\bar{R}\) is the average recruitment and \(\epsilon_i\) is the
annual deviation, which followed a normal distribution
\(N(0,\sigma_R)\).

The recruitment probability at length was assumed to be normal, i.e.,

\begin{enumerate}
\def\labelenumi{\arabic{enumi})}
\setcounter{enumi}{4}
\tightlist
\item
  \(r_l=\int_l^{l+1}\frac{1}{\sqrt{2\pi\sigma^2}} e^{(-(l-l_r)^2/2\sigma^2)}dl\)
\end{enumerate}

where \(l_r\) is the mean length at recruitment and \(\sigma^2\) is the
variance of length at recruitment.

The fishing mortality rate during the year \(i\) and length \(l\)
\((F_{i,l})\) was computed by

\begin{enumerate}
\def\labelenumi{\arabic{enumi})}
\setcounter{enumi}{5}
\tightlist
\item
  \(F_{i,l} = F_i s_l\)
\end{enumerate}

where \(F_i\) is the annual fishing mortality rate, and \(s_l\) is the
selectivity at length \(l\), which was defined by

\begin{enumerate}
\def\labelenumi{\arabic{enumi})}
\setcounter{enumi}{6}
\tightlist
\item
  \(s_l =0 \text{   if   } l < 50 \text{;  or   } s_l = 1 \text{     if      } l \geqslant 50\)
\end{enumerate}

The selectivity in Eq. 7 is a `knife-edge' function of minimum legal
size (\(lc=50\) mm.

The model for observations consisted of the total annual harvest and
total biomass in the surveys. Catch by number was estimated according to
the Baranov catch equation, i.e.,

\begin{enumerate}
\def\labelenumi{\arabic{enumi})}
\setcounter{enumi}{7}
\tightlist
\item
  \(C_{i,l}=F_{i,l} N_l (1-e^{-Z_{i,l}})/Z_{i,l}\)
\end{enumerate}

where \(C_{i,l}\) is the catch in the year \(i\) at length class \(l\).
The total annual harvest (\(Y_i\)) was estimated by:

\begin{enumerate}
\def\labelenumi{\arabic{enumi})}
\setcounter{enumi}{8}
\tightlist
\item
  \(Y_i=\sum_l W_l C_{i,l}\)
\end{enumerate}

where \(W_l\) is the average weight at length class \(l\).

Length composition in the population was estimated by:

\begin{enumerate}
\def\labelenumi{\arabic{enumi})}
\setcounter{enumi}{9}
\tightlist
\item
  \(p_{i,l}=N_{i,l}/(\sum_l N_{i,l})\)
\end{enumerate}

The population biomass at the time of the survey (within the year) was
computed by:

\begin{enumerate}
\def\labelenumi{\arabic{enumi})}
\setcounter{enumi}{10}
\tightlist
\item
  \(B_i=\psi \sum_l v_l W_l N_{i,l} e^{-\tau Z_{i,l}}\)
\end{enumerate}

where \(\psi\) is the catchability coefficient and assumed to be equal
to 0.99, \(v_l\) is the selectivity at length of the survey and assumed
to be constant and equal to 1 for all length classes, and \(\tau\) is
the time of year in which the stock assessment survey was carried out.
After that, the vulnerable biomass in the direct stock assessment
surveys (\(V_i\)) was simulated according to:

\begin{enumerate}
\def\labelenumi{\arabic{enumi})}
\setcounter{enumi}{11}
\tightlist
\item
  \(V_i = \psi \sum_l v_l W_l N_{i,l} e^{-\tau Z_{i,l}}\)
\end{enumerate}

Total biomass was computed as the sum of products between the abundance
and the average weight at length, and the spawning biomass was computed
by:

\begin{enumerate}
\def\labelenumi{\arabic{enumi})}
\setcounter{enumi}{12}
\tightlist
\item
  \(S_i=\sum_l m_l W_l N_l e^{-T_s Z_{i,l}}\)
\end{enumerate}

where \(m_l\) is the female maturity ogive, \(T_s\) is the beginning of
the spawning time within a year (set at 0.81). The model was conditioned
to the available data and known surf clam life-history parameters and
consisted of estimating the unknown parameters by fitting the population
dynamics to the data. The objective function consisted of negative
log-likelihood functions and penalized priors (Table 2 and Table 3). The
model was conditioned through an estimation procedure implemented in
ADMB (Fournier et al., 2012).

\hypertarget{section-c-simulation-of-the-management-procedure}{%
\subsection{Section C: Simulation of the management
procedure}\label{section-c-simulation-of-the-management-procedure}}

Once the OM was conditioned to the data and known life history
parameters, a forward projection phase of the population dynamics
allowed simulating the management procedure over 20 years. The
recruitment dynamics followed a Beverton and Holt stock-recruitment
relationship (SRR), described by:

\begin{enumerate}
\def\labelenumi{\arabic{enumi})}
\setcounter{enumi}{13}
\tightlist
\item
  \(R_i=\frac{4 h R_0 S_{i-1}}{(1-h) S_0+(5h-1) S_{i-1}} e^{\epsilon_i - 0.5\sigma_R^2}\)
\end{enumerate}

where \(R_0\) is the average unexploited recruitment, assumed to be
equal to the average recruitment in the period 2011-2017 (i.e.,
\(R_0=\bar{R}\)), \(S_0\) is the average unexploited spawning biomass
that produces \(R_0\), and \(h\) is the steepness (Francis, 1992; Dorn,
2012; Lee et al., 2012), which was set equal to 0.7 considering
estimates for the surf clam \emph{Spisula solidissima} (Powell et al.,
2015; Hennen et al., 2018). In Eq. 14, recruitment is a function of both
the spawning biomass and the environmental forcing, which is considered
in the simulation by allowing \(\epsilon_i\) to vary as a sequence of
switches in the operating model, i.e.

\begin{enumerate}
\def\labelenumi{\arabic{enumi})}
\setcounter{enumi}{14}
\tightlist
\item
  \(\epsilon_i=E_i e^{(\delta_i)}\)
\end{enumerate}

where \(E_i\) is the environmental forcing represented as a sequence of
switches that are alternating between two-year periods in which
recruitment is favored (\(E_i = 1\)) followed by two-year periods in
which recruitment is not favored (\(E_i = -1\)). The sequence of
switches was perturbed by stochastic annual deviations (\(\delta i\))
following a normal distribution, i.e., \(N(0,\sigma_R)\). Equation 15
allowed the simulation of future recruitment as a pattern similar to the
changes observed in the recruitment estimates obtained from the stock
assessments of 2011-2017.

The management procedure considered the current harvest rate of 25\%,
but for comparison purposes, alternative values of 0, 10, 15, 20, and
30\% were also considered. The exploitation rate \(\mu=0\) was
implemented to simulate the unexploited surf clam population as a
reference. The simulation was performed under uncertainty, sampling from
the posterior of the fitted model through Markov Chain Monte Carlo
(MCMC). The number of MCMC was obtained from 10,000 samples and saving
every 200 by using the metropolis algorithm implemented in ADMB
(Fournier et al., 2012).

\hypertarget{section-d-performance-evaluation}{%
\subsection{Section D: Performance
evaluation}\label{section-d-performance-evaluation}}

The trajectory of simulated recruitment, spawning biomass, and fishing
mortality resulting from the MP was summarized with confidence intervals
of 90\% by applying a percentile method to all realizations obtained by
MCMC. Depletion was computed as the ratio between the spawning biomass
in a given year and the average unexploited spawning biomass. Also, a
reduction of 40\% in the spawning biomass from the average unexploited
value was considered as a target reference point, i.e.,
\(S_{\text{target}} = 0.4S0\). Therefore, exploitation rates generating
reductions below the target were considered unsustainable for the surf
clam population. The probability of keeping the target was computed as
\(Pr[S_i⁄S_{\text{target}} > 1]\), whereas the probability of a collapse
was computed as \(Pr[S_i⁄S_{\text{target}} \leq 0.5]\). Exploitation
rates generating probabilities of achieving the target above 0.5 were
used as a reference for good performance.

\FloatBarrier

\hypertarget{results}{%
\section{Results}\label{results}}

\hypertarget{surf-clam-population-at-cucao-beach-and-the-operating-model}{%
\subsection{Surf clam population at Cucao beach and the operating
model}\label{surf-clam-population-at-cucao-beach-and-the-operating-model}}

In the period 2011 -- 2017, the total abundance of surf clam fluctuated
between 68 and 385 million individuals, with a mean of 174.2 million.
Total biomass ranged between 1356 and 5407 t, with a mean of 2,994 t,
whereas the vulnerable biomass fluctuated between 1261 and 5399 t, with
a mean of 2716 t (Table 1).

The operating model (OM) performed well in terms of reproducing the
observed changes in surf clam length composition (Fig. 4). The observed
length composition showed clear modal progression for sizes \(>50\) mm,
which was also shown by the fitted model (Fig. 4). According to the OM,
the mean length at recruitment (\(l_r\)) was 8.8 mm (Table 2, last
column), with specimens \(<25\) mm recruiting in 2013, 2014, and 2017
(Fig. 4). This finding provides evidence that the recruitment process in
the surf clam population of Cucao does not occur on an annual basis, but
rather with pulses of high recruitment to the population followed by
periods of lower or no recruitment, approximately every \(2-3\) years.

The population biomass showed a declining trend from 2011 to 2017 (Fig.
5A), tracking the observed total and vulnerable biomass in the surveys.
The vulnerable biomass was similar to total biomass, but the spawning
biomass was lower due to the maturity ogive and mortality prior to
spawning within the year. The average unexploited spawning biomass
(\(S_0\)) was estimated at 1,343 t, which was lower than the spawning
biomass estimated for the period 2011-2017. Hence, the target spawning
biomass for management purposes was estimated at 537 t. Recruitment was
higher in 2011-2017, with above-average values in 2013 and 2014,
followed by lower recruitment from 2015 to 2017 (Fig. 5B). The fishing
mortality rate fluctuated as the harvest but was higher in 2017 (Fig.
5C).

\hypertarget{recruitment-simulations-and-the-performance-of-the-management-procedure}{%
\subsection{Recruitment simulations and the performance of the
management
procedure}\label{recruitment-simulations-and-the-performance-of-the-management-procedure}}

According to some realizations of the OM simulations, recruitment showed
the alternating pattern between higher and lower recruitment (Fig. 6).
However, that characteristic in recruitment was obscured in the total
number of simulations, within the confidence limits of 90\% (Fig. 7A).

The spawning biomass responded to each exploitation rate (Fig. 7B), as
reflected by the approximately constant fishing mortality (Fig. 7C). The
effective catch was assumed to be identical to the quota due to rigorous
control of the harvest. Note that an exploitation rate of 30\% produces
the highest average fishing mortality, and close to that estimated in
2017 (Fig. 7C).

The current exploitation rate of 25\% produced nearly 20\% depletion in
the spawning biomass (Fig. 8), with a probability of future collapse
\textgreater{} 80\% (Fig. 9). On the other hand, an exploitation rate of
15\% kept the spawning biomass close to the target, i.e., 40\% of the
unexploited spawning biomass (Fig. 8), with probabilities \(>50\)\% once
recovered the biomass (Fig. 9). Indeed, an exploitation rate of 15\% was
able to revert the declining trend observed in the surf clam spawning
biomass (Fig. 8).

\FloatBarrier

\hypertarget{discussion}{%
\section{Discussion}\label{discussion}}

Recruitment of benthic marine invertebrates is a highly complex process
that spans a range of spatio-temporal scales (Defeo, 1996; Pineda and
Caswell, 1997) and that is modulated by environmental forcing (e.g.,
winds, waves, physiological stress) that limits larval survival and
successful settlement (Pineda, 1991; Cushing, 1995). Additionally,
density-dependent factors operating at different spatial scales are
prevalent in marine invertebrates (Hixon et al., 2012), leading to
reduced reproductive success and survival of adults (Stephens, 1999).
Adult density can positively or negatively affect recruitment success,
which then determines adult density patterns (Jenkins et al., 2009).
Both factors (i.e., environment and density-dependence) are not mutually
exclusive, but interact to determine the densities of marine benthic
populations and assemblages.

The recruitment estimates for the surf clam \emph{M. donacium} in the
period 2011-2017, and that conditioned the operating model (OM), showed
the alternation of periods with high and low recruitment in the Cucao
beach population, despite the short data series available. Two years
with high recruitment were followed by poor recruitment in 2016, after a
warm ENSO event in 2015-2016 (Jacox et al., 2016; Martínez et al.,
2017).

Recruitment failures and high temporal variability are common features
in the population dynamics of surf clams (Lima et al., 2000; Ripley and
Caswell, 2006; Aburto et al., 2013). These are general features in the
population dynamics of many species with short life cycles and can be
linked to high rates of natural mortality and greater variability in
growth rates (Bjørkvoll et al., 2012). These generalizations
notwithstanding, the estimated lifespan of the surf clam \emph{M.
donacium} at Cucao was close to 7 years, with cohorts showing a modal
progression in the size structure from 2011 to 2017. The estimated von
Bertalanffy growth parameter (\(K=0.21\) year\(^{-1}\)) indicates
theoretical longevity close to 15 years, i.e., \(t_{max} \sim 3/K\)
(Kenchington, 2014). Thus, the population's age-structure may act as a
filter of recruitment variability, dampening the effects of
environmental variability on population renewal, and hence reducing the
influence of the environment on the stock (Planque et al., 2010).

Recruitment in \emph{M. donacium} is hard to miss during the stock
assessment surveys since post-settled individuals are easily
distinguishable in the field and tend to accumulate in the swash zone
and near the mouth of estuaries or small rivers (Jaramillo et al.,
1994). Thus, as has been demonstrated in this study, the recruitment of
surf clams at Cucao beach does not always contribute noticeably to the
exploited stock biomass. Thus, settlement numbers or post-settlement
mortality, or both, may vary widely from year to year, which suggests
that environmental phenomena connected with the dispersal of larvae or
the physiology of post-settled individuals may condition the stock's
renewal. Although the information collected to date limits the
inferences that can be made about environmental phenomena that may limit
recruitment success in the surf clam \emph{M. donacium}, it is likely
that a specific combination of wave and wind conditions, at the right
time of year, is required for competent larvae to reach the shore and
settle. The total number of competent larvae that could reach the shore,
in turn, is likely to depend on advective and feeding conditions in
shelf waters during the weeks or months before the recruitment period.

It has been documented that environmental variability affects the
abundance of \emph{M. donacium} further north. For example, during the
1997-1998 El Niño, the collapse of the surf clam populations in Arica
(18º30'S) and Huasco (28º30'S) was attributed to this phenomenon, in
connection with coastal flooding and excess rainfall (Jerez et al.,
1999). In Peru, high mortality of adult \emph{M. donacium} was
attributed to the increase in temperatures caused by the 1982-1983 El
Niño (Arntz et al., 1987, 1988). Infrequent recruitment of surf clams
has also been reported previously in northern Chile, possibly in
association with environmental factors that affect the release of
gametes as well as oceanographic factors affecting the survival and
onshore supply of planktonic larvae (Thiel et al., 2007). It is common
to hear artisanal fishers talk about a ``green'' surf clam with lengths
of \(3-4\) cm that is occasionally found in the exploited banks. This is
consistent with the occasional appearance of juveniles in the annual
surveys at Cucao beach, where small individuals (lengths \(2.5-5.0\) cm)
appeared in large numbers in only one out of seven stock-assessment
surveys (2016). The inconsistent occurrence of juveniles observed in the
stock-assessment surveys was not an artifact of survey mistiming
(relative to recruitment), as indicated by the inter-annual consistency
and progression of gaps in the size-structure data collected during
surveys.

For fisheries management, the observed recruitment failures imply that,
if recruitment occurs approximately once every three years, the
exploitation rates should be lower than those recommended by the current
management procedure (\(\mu= 25\)\%), and that lower exploitation rates
(\(µ \leq 15\)\%) are needed to ensure sustainable exploitation in the
medium term. Furthermore, the current lack of knowledge on the
spatial-temporal variability of settlement and recruitment in species
such as the surf clam \emph{M. donacium} puts into question the
exploitation strategies that are currently considered as sustainable.
Typically, it is assumed that benthic species have annual recruitment,
which is not the case for \emph{M. donacium}. Therefore, this
contribution highlights an issue that warrants an even more
precautionary approach to the commercial exploitation of benthic species
with non-annual, or irregular recruitment.

A management procedure can be viewed as a ``static'' or memory-lacking
process since it does not refer to either past or future observations.
Indeed, annual quotas are computed from the standing stock assessed
directly in the field. The size-structure data are converted into age
composition data through the slicing age-class method, which is the
primary input for the yield-per-recruit model of Thompson and Bell. This
model assumes that age-classes are treated as a ``pseudo-cohorts''
without considering past recruitment to explain the current length- or
age-composition. Also, the estimation of \(F_{0.1}\) has an implicit
economic objective because it is computed from the yield-per-recruit
curve, but \(F_{0.1}\) is implicitly a function of age at first catch
and, hence, knife-edge selectivity (Deriso, 1987; Quinn and Deriso,
1999), which is probably adequate for the surf clam population.

Nevertheless, although \(F_{0.1}\) is more conservative than
\(F_{max}\), it is questionable considering the spawning potential ratio
(Shepherd, 1982; Sissenwine and Shepherd, 1987). Indeed, the realized
harvest rate associated to \(F_{0.1}\) was close to 25\% in the
management procedure and resulted in being excessive according to the
surf clam population dynamics here used as an operating model. It is
advisable to apply a harvest rate of 15\%, which may be enough to keep
the reproductive potential of the surf clam population. Furthermore, it
is advisable to implement a harvest control rule in which the harvest
rate declines when the spawning stock declines due to lower recruitment.
The ramp-like harvest control rule could be more effective for a rapid
recovery of the spawning biomass, dampening the probability of
unobserved or lower recruitments in the future. Reducing exploitation as
the stock declines results in added resilience against environmental
variability and, eventually, climate change (e.g., Merino et al., 2017).

In general, the above described ``static'' or ``memory-lacking''
procedure management is applied to almost all of the management areas
(AMEBR) in Chile, as documented in the management and exploitation plan
for each target species (Gallardo et al., 2011). Our analyses revealed
that surf clam recruitment does not occur annually, or even
periodically, with a separation of 3 or more years between high
recruitment episodes. The landing records from other areas where surf
clam populations were depleted in previous decades show that
exploitation could be unsustainable when the harvest rate intensity is
not controlled. This behavior is typical in the exploitation of surf
clam \emph{M. donacium} along the Chilean coast (e.g., Aburto and Stotz,
2013), as well as for other surf clam species (Weinberg, 1999; Laudien
et al., 2003; Fiori and Morsán, 2004; Ripley and Caswell, 2006; Herrmann
et al., 2011).

The underlying problem is that, in practice, little is known about the
intensity and success of recruitment in harvested marine populations,
which can be attributed to biases introduced by the extractive activity
itself (e.g., Punt and Cope, 2019). Sampling from the commercial catch
is usually carried out on landings, which leaves out juvenile fractions.
In the case of benthic species harvested from AMEBRs (Chile), population
surveys usually consider the fraction that can be detected visually by
scientific divers. Although this procedure includes individuals under
commercial size, it is likely to leave out newly settled individuals,
which are not always visible due to small size, pigmentation, or
behavior. Thus, the quantification of newly established fractions in
populations of commercial species usually is fraught with uncertainty
and should be approached through indirect methods. For management areas
where biomass and length-composition have been recorded annually over a
long-enough period, it is advisable to implement an integrated
stock-assessment model (Smith et al., 2009; Punt et al., 2011).
Subsequently, biological reference points should be established as a
means to assess the population's status, and to set a TAC based on
population projections. Thereafter, the AMEBR's management procedure
must be changed to keep the exploitation of benthic species within
biologically safe margins.

\hypertarget{acknowledgements}{%
\section{Acknowledgements}\label{acknowledgements}}

Partial support for LC and FJT was provided by COPAS Sur-Austral
(CONICYT PIA APOYO CCTE AFB170006). FJT also acknowledges partial
support from INCAR (CONICYT FONDAP grant 15110027). All code used to
generate this paper, as well as prior versions of this manuscript, are
available at:
\href{https://github.com/LuisACubillos/surfclampap}{github.com/LuisACubillos/surfclampap}.

\FloatBarrier

\hypertarget{references}{%
\section{References}\label{references}}

Aburto, J., Gallardo, G., Stotz, W., Cerda, C., Mondaca-Schachermayer,
C., and Vera, K. 2013. Territorial user rights for artisanal fisheries
in Chile - intended and unintended outcomes. Ocean and Coastal
Management, 71.

Aburto, J., and Stotz, W. 2013. Learning about TURFs and natural
variability: Failure of surf clam management in Chile. Ocean and Coastal
Management, 71.

Arntz, W. E., Brey, T., Tarazona, J., and Robles, A. 1987. Changes in
the structure of a shallow sandy-beach community in Peru during an el
niño event. South African Journal of Marine Science, 5: 645--658.

Arntz, W. E., Valdivia, E., and Zeballos, J. 1988. Impact of El Nino
1982-83 on the commercially exploited invertebrates (mariscos) of the
Peruvian shore. Meeresforsch., 32: 3--22.

Arntz, W. E., Gallardo, V. A., Gutiérrez, D., Isla, E., Levin, L. A.,
Mendo, J., Neira, C., et al.~2006. El Niño and similar perturbation
effects on the benthos of the Humboldt, California, and Benguela Current
upwelling ecosystems.

Berkes, F. 2003. Alternatives to conventional management: Lessons from
small-scale fisheries. Environments, 31: 5--20.

Bhaud, M. R. 1993. Relationship between larval type and geographic range
in marine species: complementary observations on gastropods.
Oceanologica Acta, 16: 191--198.

Bjørkvoll, E., Grøtan, V., Aanes, S., Sæther, B. E., Engen, S., and
Aanes, R. 2012. Stochastic population dynamics and life-history
variation in marine fish species. American Naturalist, 180: 372--387.

Botsford, L. W. 2001. Physical influences on recruitment to California
current invertebrate populations on multiple scales. In ICES Journal of
Marine Science, pp.~1081--1091.

Carstensen, D., Riascos, J. M., Heilmayer, O., Arntz, W. E., and
Laudien, J. 2010. Recurrent, thermally-induced shifts in species
distribution range in the Humboldt current upwelling system. Marine
Environmental Research, 70: 293--299. Elsevier
Ltd.~\url{http://dx.doi.org/10.1016/j.marenvres.2010.06.001}.

CCT-B, C. C. T. B. 2014. Cuota recurso macha. 1--6 pp.

Cochrane, K. L., Butterworth, D. S., De Oliveira, J. A. A., and Roel, B.
A. 1998. Management procedures in a fishery based on highly variable
stocks and with conflicting objectives: Experiences in the South African
pelagic fishery. Reviews in Fish Biology and Fisheries, 8: 177--214.

Cushing, D. H. 1995. Population Production and Regulation in the Sea: A
Fisheries Perspective. Cambridge. 368 pp.

Defeo, O. 1996. Recruitment variability in sandy beach macroinfauna:
much to learn yet. Revista chilena de historia natural, 69: 615--630.

Deriso, R. B. 1987. Optimal F0.1 criteria and their relationship to
maximum sustainable yield. Canadian Journal of Fisheries and Aquatic
Sciences, 44: 339--348.

Dorn, M. W. 2012. North American Journal of Fisheries Management Advice
on West Coast Rockfish Harvest Rates from Bayesian. North American
Journal of Fisheries Management: 37--41.

Doubleday, K. J., and Esunge, J. N. 2011. Application of Markov chains
to stock trends. Journal of Mathematics and Statistics, 7: 103--106.

Ebert, T. A. 2010. Demographic patterns of the purple sea urchin
Strongylocentrotus purpuratus along a latitudinal gradient, 1985-1987.
Marine Ecology Progress Series, 406: 105--120.

Fiori, S. M., and Morsán, E. M. 2004. Age and individual growth of
Mesodesma mactroides (Bivalvia) in the southernmost range of its
distribution. ICES Journal of Marine Science, 61: 1253--1259.

Fournier, D. A., Skaug, H. J., Ancheta, J., Ianelli, J., Magnusson, A.,
Maunder, M. N., Nielsen, A., et al.~2012. AD Model Builder: Using
automatic differentiation for statistical inference of highly
parameterized complex nonlinear models. Optimization Methods and
Software, 27: 233--249.

Francis, R. I. C. C. 1992. Use of risk analysis to assess fishery
management strategies: a case study using orange roughy (Hoplostethus
atlanticus) on the Chatham Rise, New Zealand. Canadian Journal of
Fisheries and Aquatic Sciences, 49: 922--930.

Gallardo, G. L., Stotz, W., Aburto, J., Mondaca, C., and Vera, K. 2011.
Emerging commons within artisanal fisheries. The Chilean territorial use
rights in fisheries (TURFs) within a broader coastal landscape.
International Journal of the Commons, 5: 459--484.

Gelcich, S., Hughes, T. P., Olsson, P., Folke, C., Defeo, O., Fernández,
M., Foale, S., et al.~2010. Navigating transformations in governance of
Chilean marine coastal resources. Proceedings of the National Academy of
Sciences of the United States of America, 107: 16794--16799.

Giangrande, A., Geraci, S., and Belmonte, G. 1994. Life-cycle and
life-history diversity in marine invertebrates and the implications in
community dynamics. Oceanography and marine biology: an annual review.
Vol. 32, 32: 305--333.

González, J., Stotz, W., Garrido, J., Orensanz, J. M., Parma, A. M.,
Tapia, C., and Zuleta, A. 2006. The Chilean turf system: How is it
performing in the case of the loco fishery? Bulletin of Marine Science,
78: 499--527.

Grantham, B. A., Eckert, G. L., and Shanks, A. L. 2003. Dispersal
potential of marine invertebrates in diverse habitats. Ecological
Applications, Supplement: S108--S116.

Hennen, D. R., Mann, R., Munroe, D. M., and Powell, E. N. 2018.
Biological reference points for Atlantic surfclam (Spisula solidissima)
in warming seas. Fisheries Research, 207: 126--139. Elsevier.
\url{https://doi.org/10.1016/j.fishres.2018.06.013}.

Herrmann, M., Alfaya, J. E. F., Lepore, M. L., Penchaszadeh, P. E., and
Arntz, W. E. 2011. Population structure, growth and production of the
yellow clam Mesodesma mactroides (Bivalvia: Mesodesmatidae) from a
high-energy, temperate beach in northern Argentina. Helgoland Marine
Research, 65: 285--297.

Hixon, M. A., Anderson, T. W., Buch, K. L., Johnson, D. W., Mcleod, J.
B., and Stallings, C. D. 2012. Density dependence and population
regulation in marine fish: A large-scale, long-term field manipulation.
Ecological Monographs, 82: 467--489.

Hunt, H. L., and Scheibling, R. E. 1997. Role of early post-settlement
mortality in recruitment of benthic marine invertebrates. Marine Ecology
Progress Series, 155: 269--301.

Ibarcena Fernández, W., Muñante Angulo, L., Muñante Melgar, L., and
Vasquez Flores, J. 2019. La explotación de la macha (Mesodesma donacium
Lamarck 1818) en el litoral de Tacna. Ciencia \& Desarrollo: 12--22.

Jacox, M. G., Hazen, E. L., Zaba, K. D., Rudnick, D. L., Edwards, C. A.,
Moore, A. M., and Bograd, S. J. 2016. Impacts of the 2015--2016 El Niño
on the California Current System: Early assessment and comparison to
past events. Geophysical Research Letters, 43: 7072--7080.

Jaramillo, E., Pino, M., Filun, L., and Gonzalez, M. 1994. Longshore
distribution of Mesodesma donacium (Bivalvia: Mesodesmatidae) on a sandy
beach of the south of Chile. The Veliger, 37: 192--200.

Jenkins, S. R., Marshall, D., and Fraschetti, S. 2009. Settlement and
Recruitment. In Marine Hard Bottom Communities Patterns, Dynamics,
Diversity, and Change, pp.~177--190.
\url{http://www.springerlink.com/index/10.1007/b76710}.

Jerez, G., Ariz, L., Baros, V., Olguín, A., González, J., Oliva, J.,
Ojeda, V., et al.~1999. Estudio biológico pesquero del recurso macha en
la I y III Regiones. Informe Final FIP 97-33.

Kell, L. T., Arrizabalaga, H., Merino, G., and De Bruyn, P. 2017.
Conditioning an operating model for North Atlantic Albacore. Collect.
Vol. Sci. Pap. ICCAT, 73: 1296--1327.

Kenchington, T. J. 2014. Natural mortality estimators for
information-limited fisheries. Fish and Fisheries, 15: 533--562.

Ladah, L. B., Tapia, F. J., Pineda, J., and López, M. 2005. Spatially
heterogeneous, synchronous settlement of Chthamalus spp. larvae in
northern Baja California. Marine Ecology Progress Series, 302: 177--185.

Lagos, N. A., Castilla, J. C., and Broitman, B. R. 2008. Spatial
environmental correlates of intertidal recruitment: A test using
barnacles in northern chile. Ecological Monographs, 78: 245--261.

Laudien, J., Brey, T., and Arntz, W. E. 2003. Population structure,
growth and production of the surf clam Donax serra (Bivalvia, Donacidae)
on two Namibian sandy beaches. Estuarine, Coastal and Shelf Science, 58:
105--115.

Lee, H. H., Maunder, M. N., Piner, K. R., and Methot, R. D. 2012. Can
steepness of the stock-recruitment relationship be estimated in fishery
stock assessment models? Fisheries Research, 125--126: 254--261.
Elsevier B.V. \url{http://dx.doi.org/10.1016/j.fishres.2012.03.001}.

Lima, M., Brazeiro, A., and Defeo, O. 2000. Population dynamics of the
yellow clam Mesodesma mactroides: Recruitment variability,
density-dependence and stochastic processes. Marine Ecology Progress
Series, 207: 97--108.

Marín, A., and Gelcich, S. 2012. Gobernanza y capital social en el
comanejo de recursos bentónicos en Chile: aportes del análisis de redes
al estudio de la pesca artesanal de pequeña escala. Cultura - Hombre -
Sociedad CUHSO, 22: 131--153.

Martínez, R., Zambrano, E., Nieto, J. J., Hernández, J., and Costa, F.
2017. Evolución, vulnerabilidad e impactos económicos y sociales de El
Niño 2015-2016 en América Latina. Investigaciones Geográficas: 65--78.

Matamala, M., Ther, F., Almanza, V., Bello, B., and Gutierrez, J. 2008.
Bases biológicas para la administración del recurso macha en la X
Región. Informe Final FIP 2006-26. 230 pp.

Merino, G., Arrizabalaga, H., Santiago, J., and Sharma, R. 2017. Updated
evaluation of harvest control rules for North Atlantic albacore through
management strategy evaluation. Col. Vol. Sci. Pap. ICCAT, 74: 457--478.

Mildenberger, T. K., Taylor, M. H., and Wolff, M. 2017. TropFishR: an R
package for fisheries analysis with length-frequency data.

Morgan, L. E., Botsford, L. W., Wing, S. R., and Smith, B. D. 2000.
Spatial variability in growth and mortality of the red sea urchin,
Strongylocentrotus franciscanus, in northern California. Canadian
Journal of Fisheries and Aquatic Sciences, 57: 980--992.
\url{http://www.nrcresearchpress.com/doi/abs/10.1139/f00-046}.

O'Connor, M. I., Bruno, J. F., Gaines, S. D., Halpern, B. S., Lester, S.
E., Kinlan, B. P., and Weiss, J. M. 2007. Temperature control of larval
dispersal and the implications for marine ecology, evolution, and
conservation. Proceedings of the National Academy of Sciences of the
United States of America, 104: 1266--1271.

Ospina-Alvarez, A., de Juan, S., Davis, K. J., González, C., Fernández,
M., and Navarrete, S. 2020. Integration of biophysical connectivity in
the spatial optimization of coastal ecosystem services. Science of The
Total Environment: 139367. Elsevier B.V.
\url{https://doi.org/10.1016/j.scitotenv.2020.139367}.

Ouréns, R., Freire, J., Vilar, J. A., and Fernández, L. 2014. Influence
of habitat and population density on recruitment and spatial dynamics of
the sea urchin Paracentrotus lividus: Implications for harvest refugia.
ICES Journal of Marine Science, 71: 1064--1072.

Pineda, J. 1991. Predictable Upwelling and the Shoreward Transport of
Planktonic Larvae by Internal Tidal Bores. Science, 253: 548--549.

Pineda, J. 1994. Spatial and temporal patterns in barnacle settlement
rate along a Southern California rocky shore. Marine Ecology Progress
Series, 107: 125--138.

Pineda, J., and Caswell, H. 1997. Dependence of settlement rate on
suitable substrate area. Marine Biology, 129: 541--548.

Pineda, J. 2000. Linking larval settlement to larval transport:
assumptions, potentials and pitfalls. Oceanography of the Eastern
Pacific: 84--105.

Pineda, J., Reyns, N. B., and Starczak, V. R. 2009. Complexity and
simplification in understanding recruitment in benthic populations.

Planque, B., Fromentin, J. M., Cury, P., Drinkwater, K. F., Jennings,
S., Perry, R. I., and Kifani, S. 2010. How does fishing alter marine
populations and ecosystems sensitivity to climate? Elsevier B.V.
\url{http://dx.doi.org/10.1016/j.jmarsys.2008.12.018}.

Powell, E. N., Klinck, J. M., Munroe, D. M., Hofmann, E. E., Moreno, P.,
and Mann, R. 2015. The value of captains' behavioral choices in the
success of the surfclam (Spisula solidissima) fishery on the U.S.
mid-atlantic coast: A model evaluation. Journal of Northwest Atlantic
Fishery Science, 47: 1--27.

Punt, A. E., Smith, D. C., and Smith, A. D. M. 2011. Among-stock
comparisons for improving stock assessments of data-poor stocks: The
`robin Hood' approach. ICES Journal of Marine Science, 68: 972--981.

Punt, A. E., Huang, T., and Maunder, M. N. 2013. Review of integrated
size-structured models for stock assessment of hard-to-age crustacean
and mollusc species. ICES Journal of Marine Science, 70: 16--33.

Punt, A. E., Butterworth, D. S., de Moor, C. L., De Oliveira, J. A. A.,
and Haddon, M. 2016. Management strategy evaluation: Best practices.
Fish and Fisheries, 17: 303--334.

Punt, A. E., and Cope, J. M. 2019. Extending integrated stock assessment
models to use non-depensatory three-parameter stock-recruitment
relationships. Fisheries Research, 217: 46--57. Elsevier.
\url{http://dx.doi.org/10.1016/j.fishres.2017.07.007}.

Quinn, T. J., and Deriso, R. B. 1999. Quantitative Fish Dynamics. Oxford
University Press. 560 pp.

Riascos, J. M., Carstensen, D., Laudien, J., Arntz, W. E., Oliva, M. E.,
Guntner, A., and Heilmayer, O. 2009. Thriving and declining: Climate
variability shaping life-history and population persistence of Mesodesma
donacium in the Humboldt Upwelling System. Marine Ecology Progress
Series, 385: 151--163.

Ripley, B. J., and Caswell, H. 2006. Recruitment variability and
stochastic population growth of the soft-shell clam , Mya arenaria, 193:
517--530.

Rubilar, P., Ariz, L., Ojeda, V., Lozada, E., Campos, P., Jerez, G.,
Osorio, C., et al.~2001. Estudio biológico pesquero del recurso macha en
la X Región. Informe Final FIP 2000-17. 242 pp.

Shanks, A. L., and Shearman, R. K. 2009. Paradigm lost? Cross-shelf
distributions of intertidal invertebrate larvae are unaffected by
upwelling or downwelling. Marine Ecology Progress Series, 385: 189--204.

Shepherd, J. G. 1982. A Versatile New Stock-Recruitment Relationship for
Fisheries, and the Construction of Sustainable Yield Curves. ICES
Journal of Marine Science, 40: 67--75.

Sissenwine, M. P., and Shepherd, J. G. 1987. An Alternative Perspective
on Recruitment Overfishing and Biological Reference Points. Canadian
Journal of Fisheries and Aquatic Sciences, 44: 913--918.

Smith, D., Punt, A., Dowling, N., Smith, A., Tuck, G., and Knuckey, I.
2009. Reconciling Approaches to the Assessment and Management of
Data-Poor Species and Fisheries with Australia's Harvest Strategy
Policy. Marine and Coastal Fisheries, 1: 244--254.

Starr, P. J., Breen, P. A., Hilborn, R. H., and Kendrick, T. H. 1997.
Evaluation of a management decision rule for a New Zealand rock lobster
substock. In Marine and Freshwater Research, pp.~1093--1101.

Stephens, P. A. . S. W. J. . F. R. P. 1999. What is the Allee effect?
Oikos, 87: 185--190.

Stotz, W., Lancellotti, D. A., Lohrmann, K., von Brand, E., Aburto, J.,
Caillaux, L. M., Valdebenito, M., et al.~2003. Repoblamiento de bancos
de macha en playa `Las machas' de Arica, I Región. Informe Final FIP
2001-24. 207 pp.

Sullivan, P. J., Han-Lin Lai, and Gallucci, V. F. 1990. A
catch-at-length analysis that incorporates a stochastic model of growth.
Canadian Journal of Fisheries and Aquatic Sciences, 47: 184--198.

Tapia, F. J., and Pineda, J. 2007. Stage-specific distribution of
barnacle larvae in nearshore waters: Potential for limited dispersal and
high mortality rates. Marine Ecology Progress Series, 342: 177--190.

Thiel, M., Macaya, E. C., Acuña, E., Arntz, W. E., Bastias, H.,
Brokordt, K., Camus, P. A., et al.~2007. The Humboldt Current System of
northern and central Chile. Oceanography and Marine Biology Vol 45, 45:
195--344. \url{http://www.vliz.be/vmdcdata/Imis2/ref.php?refid=111470}.

Weinberg, J. R. 1999. Age-structure, recruitment, and adult mortality in
populations of the Atlantic surfclam, Spisula solidissima, from 1978 to
1997. Marine Biology, 134: 113--125.

\hypertarget{caption-of-figures}{%
\section{Caption of figures}\label{caption-of-figures}}

Figure 1. Principal landing points of surf clam \emph{M. donacium} along
the Chilean coast (left), and performance of regional landings from 2000
(right). Source: SERNAPESCA, Servicio Nacional de Pesca y Acuicultura,
Chile.

Figure 2. General simulation procedure implemented for the surf clam
\emph{M. donacium} in the AMEBR Cucao.

Figure 3. Current management procedure for the surf clam \emph{M.
donacium} in the AMEBR Cucao.

Figure 4. Observed and predicted length composition of surf clam
\emph{M. donacium} at Cucao in 2011-2017. The predicted length
composition comes from the conditioned operating model.

Figure 5. Population biomasses and catch (A), annual recruitment (B),
and fishing mortality rate (C) of surf clam \emph{M. donacium} at Cucao
during 2011-2017 obtained from the conditioned operating model.

Figure 6. Single realizations of simulated future recruitment for surf
clam \emph{M. donacium} using 8 different harvest rates.

Figure 7. Summary of 500 simulations of projected recruitment (A), and
responses in the spawning biomass (B) and fishing mortality (C) of the
surf clam \emph{M. donacium} at Cucao beach using 6 different (and
constant) exploitation rates. Light purple shading indicates observed
data from 2011 to 2017. Gray shading corresponds to 90\% confidence
limits for projected variables. The dashed horizontal line is the target
spawning stock biomass.

Figure 8. Expected depletion of the spawning biomass of surf clam
\emph{M. donacium} at Cucao beach according to six different
exploitation rates. Light purple shading indicates observed data from
2010 to 2017. Gray shading corresponds to 90\% confidence intervals for
expected depletion. The dashed horizontal line is the target depletion.

Figure 9. Probability of collapse (A) and the probability of achieving
the target biomass (B) of 40\% surf clam spawning biomass at Cucao beach
under different exploitation rates (colored lines).

\end{document}
