\PassOptionsToPackage{unicode=true}{hyperref} % options for packages loaded elsewhere
\PassOptionsToPackage{hyphens}{url}
%
\documentclass[]{article}
\usepackage{lmodern}
\usepackage{amssymb,amsmath}
\usepackage{ifxetex,ifluatex}
\usepackage{fixltx2e} % provides \textsubscript
\ifnum 0\ifxetex 1\fi\ifluatex 1\fi=0 % if pdftex
  \usepackage[T1]{fontenc}
  \usepackage[utf8]{inputenc}
  \usepackage{textcomp} % provides euro and other symbols
\else % if luatex or xelatex
  \usepackage{unicode-math}
  \defaultfontfeatures{Ligatures=TeX,Scale=MatchLowercase}
\fi
% use upquote if available, for straight quotes in verbatim environments
\IfFileExists{upquote.sty}{\usepackage{upquote}}{}
% use microtype if available
\IfFileExists{microtype.sty}{%
\usepackage[]{microtype}
\UseMicrotypeSet[protrusion]{basicmath} % disable protrusion for tt fonts
}{}
\IfFileExists{parskip.sty}{%
\usepackage{parskip}
}{% else
\setlength{\parindent}{0pt}
\setlength{\parskip}{6pt plus 2pt minus 1pt}
}
\usepackage{hyperref}
\hypersetup{
            pdfborder={0 0 0},
            breaklinks=true}
\urlstyle{same}  % don't use monospace font for urls
\usepackage[margin=1in]{geometry}
\usepackage{graphicx,grffile}
\makeatletter
\def\maxwidth{\ifdim\Gin@nat@width>\linewidth\linewidth\else\Gin@nat@width\fi}
\def\maxheight{\ifdim\Gin@nat@height>\textheight\textheight\else\Gin@nat@height\fi}
\makeatother
% Scale images if necessary, so that they will not overflow the page
% margins by default, and it is still possible to overwrite the defaults
% using explicit options in \includegraphics[width, height, ...]{}
\setkeys{Gin}{width=\maxwidth,height=\maxheight,keepaspectratio}
\setlength{\emergencystretch}{3em}  % prevent overfull lines
\providecommand{\tightlist}{%
  \setlength{\itemsep}{0pt}\setlength{\parskip}{0pt}}
\setcounter{secnumdepth}{0}
% Redefines (sub)paragraphs to behave more like sections
\ifx\paragraph\undefined\else
\let\oldparagraph\paragraph
\renewcommand{\paragraph}[1]{\oldparagraph{#1}\mbox{}}
\fi
\ifx\subparagraph\undefined\else
\let\oldsubparagraph\subparagraph
\renewcommand{\subparagraph}[1]{\oldsubparagraph{#1}\mbox{}}
\fi

% set default figure placement to htbp
\makeatletter
\def\fps@figure{htbp}
\makeatother


\author{}
\date{\vspace{-2.5em}}

\begin{document}

\hypertarget{materials-and-methods}{%
\section{Materials and methods}\label{materials-and-methods}}

\hypertarget{study-area-and-data-sources}{%
\subsection{Study area and data
sources}\label{study-area-and-data-sources}}

The study area is Cucao beach (24º36'S-74º08'W), located on the western
shore of Chiloé island, northern Chilean Patagonia (Fig. 1). Over the
past decade, Cucao has been one of the main harvesting areas for the
surf clam M. donacium, with three organizations of artisanal fishers
having territorial use rights since 2015 (Fig. 1). The data were
obtained from six stock assessment surveys carried out between 2011 and
2017 (Table 1). Before 2015, the stock assessments of surf clam were
carried out to establish annual catch quotas. Since 2015, data from the
assessments became input information for harvesting surf clam under the
AMEBR management procedure.

\hypertarget{evaluation-of-the-management-procedure}{%
\subsection{Evaluation of the management
procedure}\label{evaluation-of-the-management-procedure}}

A simulation was implemented to evaluate the performance of the
management procedure (MP) for M. donacium in Cucao. The simulation
involved the steps of the Management Strategy Evaluation (MSE) framework
(Starr et al., 1997; Cochrane et al., 1998; Punt et al., 2016; Kell et
al., 2017). In this framework, one of the steps is conditioning an
Operating Model (OM) based on data and knowledge for the surf clam
population dynamics. The OM allowed us to evaluate the MP under
uncertainty (Fig. 2), especially in terms of recruitment, which during
the studied period exhibited pulses of high recruitment followed by
years of low to nil recruitment. The OM allowed simulating the perceived
vulnerable biomass in the stock assessment surveys for a window of 20
years into the future, along with the quota and the realized total
harvest under a constant harvest rate strategy.

The simulation modeling to evaluate the MP for surf clam consisted of
the steps described in the following sections: Section A describes the
current MP for surf clam in Cucao. Section B describes the OM that
specifies the true structure and processes modulating the surf clam
population dynamics, with emphasis on conditioning the OM to the
available data and knowledge (Kell et al., 2017). Section C describes
the phase of projecting the operating model 20 years into the future.
For each year, the OM provides a population that can be sampled in a way
similar to the stock assessment surveys carried out in the field. The
projected OM included the recruitment dynamics and its response to
fishing and environmental forcing. Section D describes the statistics
used to summarize the performance of the current and alternative
management procedures for surf clam in Cucao.

\hypertarget{section-a-the-management-procedure-for-surf-clam}{%
\subsection{Section A: The management procedure for surf
clam}\label{section-a-the-management-procedure-for-surf-clam}}

The management procedure corresponding to the Cucao AMEBR is shown in
Figure 3. A team of technicians and professional divers carry out a
stock assessment survey annually (see Table 1). The survey is designed
to provide estimates of total abundance and biomass in the surveyed
area. The estimate of biomass is size-structured, allowing the
estimation of vulnerable biomass, which is defined by surf clams larger
than 50 mm length (i.e., the minimum legal size). The stock assessment
team computes yield per recruit using a Thomson and Bell model, and then
F0.1 (Deriso, 1987) to compute the quota to be harvested. Nevertheless,
in practical terms, the harvest decision has resulted in a constant
Quota/Vulnerable biomass ratio of approximately 25\% (Table 1). Thus,
the current management procedure can be simplified by formulating the
following empirical harvest control rule:

\begin{enumerate}
\def\labelenumi{\arabic{enumi})}
\tightlist
\item
  \(Q_i=0.25 V_i\)
\end{enumerate}

where \(V_i\) is the survey estimate for vulnerable biomass in the i-th
year and \(Q_i\) is the quota of surf clam requested by the fishers
organizations to the centralized management agency, i.e., the
Undersecretariat of Fisheries and Aquaculture (SUBPESCA), which reviews
the technical reports and approves the harvest quotas. The management
procedure is essentially empirical since it uses the vulnerable biomass
estimated in the survey as an indicator of the surf clam status, and the
primary input to the harvest control rule (Table 1). Once SUBPESCA
approves the quota, fishers can harvest the surf clam from the
management area. At the time of harvest, catches are monitored and
logged by the Chilean National Fisheries Service (SERNAPESCA).

\hypertarget{section-b-the-operating-model}{%
\subsection{Section B: The operating
model}\label{section-b-the-operating-model}}

The operating model (OM) was conditioned to know life-history parameters
of surf clam and total biomass and population size-structure data
obtained from the direct stock-assessment surveys (Table 1 and Table 2).
The OM was based on an integrative size-structured stock assessment
model (Sullivan et al., 1990; Punt et al., 2013), expressed by

\begin{enumerate}
\def\labelenumi{\arabic{enumi})}
\setcounter{enumi}{1}
\tightlist
\item
  \(N_{i,l}=G_{l,l'} N_{i-1,l} e^{-Z_{i-1,l} }+ r_l R_i\)
\end{enumerate}

where \(N_{i,l}\) is the abundance of length-class \(l\) at the
beginning of year \(i\), \(Z\) is the instantaneous total mortality
rate, i.e., \(Z=F+M\), where \(F\) is the fishing mortality, and M is
the natural mortality rate (set equal to 0.3). \(R_i\) is recruitment,
\(r_l\) is the distribution of recruitment by length-classes, and
\(G_{l,l'}\) is a growth transition matrix described by

\begin{enumerate}
\def\labelenumi{\arabic{enumi})}
\setcounter{enumi}{2}
\tightlist
\item
  \(G_{l,l'}=\int_l^{l'} (l'-l)^{\alpha_j} e^{-(l'-l)/\beta_p } dl/\beta_p\)
\end{enumerate}

where \(l\) is the length class, and \(\alpha_j\) and \(\beta_p\) are
parameters describing a gamma probability function. Recruitment was
estimated according to:

\begin{enumerate}
\def\labelenumi{\arabic{enumi})}
\setcounter{enumi}{3}
\tightlist
\item
  \(R_i = \bar{R}e^{\epsilon_i}\)
\end{enumerate}

where \(\bar{R}\) is the average recruitment and \(\epsilon_i\) is the
annual deviation, which followed a normal distribution
\(N(0,\sigma_R)\).

The recruitment probability at length was assumed to be normal, i.e.,

\begin{enumerate}
\def\labelenumi{\arabic{enumi})}
\setcounter{enumi}{4}
\tightlist
\item
  \(r_l=\int_l^{l+1}\frac{1}{\sqrt{2\pi\sigma^2}} e^{(-(l-l_r)^2/2\sigma^2)}dl\)
\end{enumerate}

where \(l_r\) is the mean length at recruitment and \(\sigma^2\) is the
variance of length at recruitment.

The fishing mortality rate during the year \(i\) and length \(l\)
\((F_{i,l})\) was computed by

\begin{enumerate}
\def\labelenumi{\arabic{enumi})}
\setcounter{enumi}{5}
\tightlist
\item
  \(F_{i,l} = F_i s_l\)
\end{enumerate}

where \(F_i\) is the annual fishing mortality rate, and \(s_l\) is the
selectivity at length \(l\), which was defined by

\begin{enumerate}
\def\labelenumi{\arabic{enumi})}
\setcounter{enumi}{6}
\tightlist
\item
  \(s_l =0 \text{   if   } l < 50 \text{;  or   } s_l = 1 \text{     if      } l \geqslant 50\)
\end{enumerate}

The selectivity in Eq. 7 is a `knife-edge' function of minimum legal
size (\(lc=50\) mm.

The model for observations consisted of the total annual harvest and
total biomass in the surveys. Catch by number was estimated according to
the Baranov catch equation, i.e.,

\begin{enumerate}
\def\labelenumi{\arabic{enumi})}
\setcounter{enumi}{7}
\tightlist
\item
  \(C_{i,l}=F_{i,l} N_l (1-e^{-Z_{i,l}})/Z_{i,l}\)
\end{enumerate}

where \(C_{i,l}\) is the catch in the year \(i\) at length class \(l\).
The total annual harvest (\(Y_i\)) was estimated by:

\begin{enumerate}
\def\labelenumi{\arabic{enumi})}
\setcounter{enumi}{8}
\tightlist
\item
  \(Y_i=\sum_l W_l C_{i,l}\)
\end{enumerate}

where \(W_l\) is the average weight at length class \(l\).

Length composition in the population was estimated by:

\begin{enumerate}
\def\labelenumi{\arabic{enumi})}
\setcounter{enumi}{9}
\tightlist
\item
  \(p_{i,l}=N_{i,l}/(\sum_l N_{i,l})\)
\end{enumerate}

The population biomass at the time of the survey (within the year) was
computed by:

\begin{enumerate}
\def\labelenumi{\arabic{enumi})}
\setcounter{enumi}{10}
\tightlist
\item
  \(B_i=\psi \sum_l v_l W_l N_{i,l} e^{-\tau Z_{i,l}}\)
\end{enumerate}

where \(\psi\) is the catchability coefficient and assumed to be equal
to 0.99, \(v_l\) is the selectivity at length of the survey and assumed
to be constant and equal to 1 for all length classes, and \(\tau\) is
the time of year in which the stock assessment survey was carried out.
After that, the vulnerable biomass in the direct stock assessment
surveys (\(V_i\)) was simulated according to:

\begin{enumerate}
\def\labelenumi{\arabic{enumi})}
\setcounter{enumi}{11}
\tightlist
\item
  \(V_i = \psi \sum_l v_l W_l N_{i,l} e^{-\tau Z_{i,l}}\)
\end{enumerate}

Total biomass was computed as the sum of products between the population
and the average weight at length, and the spawning biomass was computed
by:

\begin{enumerate}
\def\labelenumi{\arabic{enumi})}
\setcounter{enumi}{12}
\tightlist
\item
  \(S_i=\sum_l m_l W_l N_l e^{-T_s Z_{i,l}}\)
\end{enumerate}

where \(m_l\) is the female maturity ogive, \(T_s\) is the beginning of
the spawning time within a year (set at 0.81). The model was conditioned
to the available data and known surf clam life-history parameters and
consisted of estimating the unknown parameters by fitting the population
dynamics to the data. The objective function consisted of negative
log-likelihood functions and penalized priors (Table 2 and Table 3). The
model was conditioned through an estimation procedure implemented in
ADMB (Fournier et al., 2012).

\hypertarget{section-c-simulation-of-the-management-procedure}{%
\subsection{Section C: Simulation of the management
procedure}\label{section-c-simulation-of-the-management-procedure}}

Once the OM was conditioned to the data and known life history
parameters, a forward projection phase of the population dynamics
allowed simulating the management procedure over 20 years. The
recruitment dynamics followed a Beverton and Holt stock-recruitment
relationship (SRR), described by:

\begin{enumerate}
\def\labelenumi{\arabic{enumi})}
\setcounter{enumi}{13}
\tightlist
\item
  \(R_i=\frac{4 h R_0 S_{i-1}}{(1-h) S_0+(5h-1) S_{i-1}} e^{\epsilon_i - 0.5\sigma_R^2}\)
\end{enumerate}

where \(R_0\) is the average unexploited recruitment, assumed to be
equal to the average recruitment in the period 2011-2017 (i.e.,
\(R_0=\bar{R}\)), \(S_0\) is the average unexploited spawning biomass
that produces \(R_0\), and \(h\) is the steepness (Francis, 1992; Dorn,
2012; Lee et al., 2012), which was set equal to 0.7 considering
estimates for the surf clam Spisula solidissima (Powell et al., 2015;
Hennen et al., 2018). In Eq. 14, recruitment is a function of both the
spawning biomass and the environmental forcing, which is considered in
the simulation by allowing \(\epsilon_i\) to vary as a sequence of
switches in the operating model, i.e.

\begin{enumerate}
\def\labelenumi{\arabic{enumi})}
\setcounter{enumi}{14}
\tightlist
\item
  \(\epsilon_i=E_i e^{(δ_i)}\)
\end{enumerate}

where \(E_i\) is the environmental forcing represented as a sequence of
switches that are alternating between two-year periods in which
recruitment is favored (\(E_i = 1\)) followed by two-year periods in
which recruitment is not favored (\(E_i = -1\)). The sequence of
switches was perturbed by stochastic annual deviations (\(\delta i\))
following a normal distribution, i.e., \(N(0,\sigma_R)\). Equation 15
allowed the simulation of future recruitment as a pattern similar to the
changes observed in the recruitment estimates obtained from the stock
assessments of 2011-2017.

The management procedure considered the current harvest rate of 25\%,
but for comparison purposes, alternative values of 0, 10, 15, 20, and
30\% were also considered. The exploitation rate \(\mu=0\) was
implemented to simulate the unexploited surf clam population as a
reference. The simulation was performed under uncertainty, sampling from
the posterior of the fitted model through Markov Chain Monte Carlo
(MCMC). The number of MCMC was obtained from 10,000 samples and saving
every 200 by using the metropolis algorithm implemented in ADMB
(Fournier et al., 2012).

\hypertarget{section-d-performance-evaluation}{%
\subsection{Section D: Performance
evaluation}\label{section-d-performance-evaluation}}

The trajectory of simulated recruitment, spawning biomass, and fishing
mortality resulting from the MP was summarized with confidence intervals
of 90\% by applying a percentile method to all realizations obtained by
MCMC. Depletion was computed as the ratio between the spawning biomass
in a given year and the average unexploited spawning biomass. Also, a
reduction of 40\% in the spawning biomass from the average unexploited
value was considered as a target reference point, i.e.,
\(S_{\text{target}} = 0.4S0\). Therefore, exploitation rates generating
reductions below the target were considered unsustainable for the surf
clam population. The probability of keeping the target was computed as
\(Pr[S_i⁄S_{\text{target}} >1]\), whereas the probability of a collapse
was computed as \(Pr[S_i⁄S_{\text{target}} \leq 0.5]\). Exploitation
rates generating probabilities of achieving the target above 0.5 were
used as a reference for good performance.

\end{document}
